\chapter{Analysis}\label{chap1}
This chapter describes the analysis methodology of using $Z\to\tau\tau$ events to measure Monte Carlo correction factors for tau identification algorithms on the high-$p_T$ region.


\section{$Z\to\tauh\taul$ tag and probe study}
As we saw in the previous section different working points are defined for RNN score relative to the efficiency of selecting true $\tauh$ candidates. When the efficiency of the working points is measured in data and simulation, a correction factor is derived and then applied to the simulation in order for the signal efficiency to agree between data and simulation REF TAU ID PERFORMANCE. Because of the top quark mass, $t\bar{t}$ events are used as a source of high momentum taus for measuring correction factors on the high-$p_T$ bins. But as we already saw in section \ref{chap2sec2}, LU may not hold on W decays. For that reason our study is aimed to use $Z\to\tau\tau$ events for deriving simulation correction factors on the high-$p_T$ region.   

\section{Monte Carlo Samples}

\section{The Collinear Approximation}

\section{Event Selection}