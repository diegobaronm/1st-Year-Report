\chapter{Results}\label{chap5}
In this section we present the preliminary results of the performance of the $\tauh$ identification algorithm using $\int\mathcal{L} dt=139.2$ fb$^{-1}$ of data recorded between 2015 and 2018.
The correction factors are applied to simulation in order to match the efficiency observed in data. These correction factors are defined as the ratio between the efficiency measured in data and in simulation. For the present report, we present the value of the correction factors for \textit{Tight} ID working point for $\tauh$ candidates with $\pt$ above 45 GeV.

\section{$\mu\tau$ Final state}
We define the simulation correction factor as
\begin{equation}
	C_{\text{Tight-ID}}=\frac{\mathcal{E}_{\text{Data}}}{\mathcal{E}_{\text{MC}}}.
\end{equation}
The value obtained for the final state that contains one muon and a $\tauh$ candidate is

All the distributions of the relevant cuts for selecting our signal events after applying all the other cuts are shown in Appendix A.
\section{$e\tau$ Final state}
The value obtained for $C_{\text{Tight-ID}}$ in the final state that contains one electron and a $\tauh$ candidate is

All the distributions of the relevant cuts for selecting our signal events after applying all the other cuts are shown in Appendix A.
\section{Discussion}

