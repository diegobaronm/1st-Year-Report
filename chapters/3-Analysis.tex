\chapter{Analysis}\label{chap1}
At the begining of this chapter, a review of the ATLAS detector at the LHC and a description of the reconstruction and identification of hadronic Tau decays on ATLAS are discussed. The last sections are devoted to describe the Analysis methodology of using $Z\to\tau\tau$ events to measure Monte Carlo correction factors for Tau identification algorithms on the high-$p_T$ region.

\section{The LHC and the ATLAS experiment}

\section{Tau Reconstruction and Identification on the ATLAS detector}
Leptonically decaying taus ($\tau_\text{lep}$), may have a higher impact parameter and tend to have a softer $p_T$ spectrum compared with prompt W or Z boson decays to muons or electrons. This variables are not sufficient in principle to differentiate between $\taul$ and prompt muons or electrons. In the case of hadronically decaying taus ($\tau_\text{had}$), as we will see, there are a lot more variables we could use to tag the presence of a $\tauh$.

As we saw in section \ref{chap2sec1}, $\tauh$ decays can be classified in 1-prong or 3-prong, depending on the number of charged particles in the decay. A detailed review of the reconstruction procedure is discussed on \cite{Aad:2014rga}. $\tauh$ candidates are seeded by jets using the anti-$k_t$ algorithm \cite{Cacciari:2008gp}, with a distance parameter of 0.4. Jets are required to have $p_T>10$ GeV and $|\eta|<2.5$. Candidates between the barrel and forward calorimeter ($1.37<|\eta|<1.52$) are excluded due to poor instrumentation in this region.

The axis of the seed jet is defined by the energy-weighted barycentre of all clusters of calorimeter cells, called \textit{TopoClusters} \cite{Aad:2016upy}. The $\tauh$ vertex is defined as the vertex with the highest $p_T$-weighted fraction of all tracks with $p_T>0.5$ GeV within a cone of $R=0.2$ around the seed jet axis. Tracks within a cone of $R=0.4$ are classified with a set of boosted decision trees (BDTs) into core and isolation tracks, the number of core tracks defines the number of prongs. Candidates with neither one or three tracks are rejected. Additionally, the sum of the charge of the tracks is required to be $\pm 1$.     

The tau reconstruction algorithm does not provide discrimination against jets that could mimic the signal of a $\tauh$ decay in the detector. Therefore, algorithms that perform this task have been developed. Previously, a BDT was used to discriminate jets against $\tauh$. Recently, a recurrent neural network (RNN) classifier that provides improved performance than the BDTs is in use \cite{Deutsch:2680523}.

The RNN makes use of a set of variables like: $\tauh$ track features, information about energy deposits on the calorimeters clusters and high level features like the mass of the $\tauh$ candidate tracks. This variables are used to exploit the differences in the shapes of the showers between $\tauh$ and jets. In general, $\tauh$ showers tend to be more collimated and to have fewer tracks than jets. A representation of this is shown in Fig.\ref{Fig4}. 

\begin{figure}[h]
	\centering
	\includegraphics[width=0.7\textwidth]{figures/Fig4}
	\caption{Graphic representation that shows the main differences between a 3-prong $\tauh$ and a jet originated from quark or gluon radiation (QCD jets). Charged hadrons are shown as thick lines and dashed lines represent neutral particles. The green cone is drawn to depict how $\tauh$ product decays are more collimated.}
	\label{Fig4}
\end{figure}
Separated algorithms are trained for 1-prong and 3-prongs. The final RNN score assigned to each event corresponds to the fraction of rejected true $\tauh$, independent of $p_T$ and number of interaction per bunch crossing (pileup). Four working points with increasing background rejection are defined to be used in physics analysis. The working points and background rejection factors are shown in Table \ref{Table2}. A plot comparing the true $\tauh$ selection efficiency versus the background rejection power for the RNN and BDT algorithm is shown in Fig. , the performance of the RNN is better than the BDT classifier. Finally, the distribution for the RNN score for true and fake $\tauh$ is shown in Fig. , for both 1-prong and 3-prong decays.
\begin{table}[]
	\begin{tabular}{|l|l|l|l|l|l|l|}
		\hline
		Working point & \multicolumn{2}{l|}{Singal efficiency (\%)} & \multicolumn{2}{l|}{BG rejection BDT} & \multicolumn{2}{l|}{BG rejection RNN} \\ \hline
		& 1-prong              & 3-prong              & 1-prong               & 3-prong               & 1-prong               & 3-prong               \\ \hline
		Tight         & 60                   & 45                   & 40                    & 400                   & 70                    & 700                   \\ \hline
		Medium        & 75                   & 60                   & 20                    & 150                   & 35                    & 240                   \\ \hline
		Loose         & 85                   & 75                   & 12                    & 61                    & 21                    & 90                    \\ \hline
		Very loose    & 95                   & 95                   & 5.3                   & 11.2                  & 9.9                   & 16                    \\ \hline
	\end{tabular}
	\caption{Working points with their corresponding true $\tauh$ selection efficiency and the background rejection factors. The scores are shown for both RNN and BDT algorithms.}
	\label{Table2}
\end{table}
\begin{figure}[h]
	\centering
	\includegraphics[width=0.7\textwidth]{figures/Fig5}
	\caption{Comparisson between the performance of BDT and RNN algorithms. Working points are shown as points. Notice there is a trade off between true $\tauh$ efficiency and background rejection power. Taken from \cite{Deutsch:2680523}}
	\label{Fig5}
\end{figure}
\begin{figure}[h]
	\centering
	\includegraphics[width=1\textwidth]{figures/Fig6}
	\caption{Distribution of the RNN scores for true and fake $\tauh$ candidates for 1-prong (a) and 3-prong (b) cases. Taken from \cite{Deutsch:2680523}.}
	\label{Fig6}
\end{figure}
\section{Monte Carlo Samples}

\section{The Collinear Approximation}

\section{Event Selection}