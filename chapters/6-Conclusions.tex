\chapter{Conclusions and prospects}\label{chap6}
The study presented is a work in progress. Given that the jets produced in hadronic tau decays have particular features, the ATLAS colaboration has developed algorithms to help with the identification of these decays. Simulation correction factors account for the difference in the efficiency of these algorithms between real data and simulation. The aim of our study is to derive the correction factors using $Z\to\tauh l=e,\mu$ events in a particular phase space where the $\tauh$ are boosted in the transverse plane. A preliminary result of the value of the correction factor for the tight-ID working point is presented in this work.

The generators used for our study tend to underestimate the Z$(\pt)$ in the high momentum region. To correct for this effect we plan to reweight the MC predictions. This will offer us the possibility to derive the corrections factors for both generators and also claim a systematic uncertainty accounting for the Z$(\pt)$ modelling. Furthermore, in the future we will have to extend our study to the looser tau-ID working points. In this region the challenge will be to control the contribution coming from the MJ background. 

Finally, we hope that all we have learnt about the highly boosted di-tau systems can be applied to a future analysis.  The goal of this study will be to observe vector boson fusion production in final states with tau leptons. In this case the di-tau systems will recoil against the jets produced in the event. More information about previous observations of this topic can be find in \cite{Aad:2014dta,Aaboud:2017emo}. 

