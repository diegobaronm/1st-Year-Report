\chapter*{Abstract}
This report is a review of the work done by the student Diego Baron during his first year in the PhD in particle physics program at the University of Manchester. Thus, it is a work in progress. The tau lepton has life time of the order of picoseconds, thus it decays before it can reach the ATLAS detector. So the indirect observation of the tau is done by measuring its decay products. The tau lepton has enough mass to decay not only into the lighter leptons but into hadrons. The leptonic tau decays at the moment can not be differentiated from prompt leptons. In the case of hadronic tau decays the jets produced have different characteristics that make them distinguishable from QCD initiated jets. Different algorithms have been trained to separate true hadronically decaying taus from QCD jets. The efficiency of these algorithms is compared in simulation and data and correction factors are derived to account for the differences that may arise from the simulation limitations. Our study makes use of $Z\to\tauh l=e,\mu$ events, highly boosted in the transverse plane, to evaluate the performance of the tau-ID algorithms for high-$\pt$ taus.



