\chapter{Tau physics overview}\label{chap:relatedwork}
This chapter is a review of the Tau lepton properties. They include the nature of this particle, its interactions with other Standard Model (SM) particles, its main decay modes and the physics implications of the so called Lepton Universality (LU), one of the SM predictions.  

\section{The Tau Lepton}
The Tau is a spin-$\frac{1}{2}$, electrically charged particle that belongs to the same family of particles as the electron, the muon and the neutrinos, they are all called \textit{]leptons}. Leptons are elementary particles that interact only via the weak and electromagnetic interactions, for the latter case only if they have electric charge.  

The first hints for the tau existence came from experiments conducted at the Stanford Linear Accelerator Center and Lawrence Berkeley National Laboratory \cite{PhysRevLett.35.1489}. They discovered 64 events of the form:
\begin{equation}
	e^+ + e^- \to e^\pm + \mu^\mp + \geq \text{2 undetected particles},
\end{equation}
for which there was no conventional explanation at that time. Later on, it was discovered that these events came from the production of a pair of tau particles and then a its subsequent decay on one electron, a muon and four neutrinos. Events like,
\begin{equation}
e^+ + e^- \to \tau^+ \tau^- \to e^\pm + \mu^\mp + 4\nu,
\end{equation}	
were later explored to derive tau mass and spin, confirming the existence of a third generation of leptons. 

The tau mass being $1776.86 \pm 0.12$ MeV allows this lepton not only to decay into the other lighter lepton generations (\textit{leptonic tau decays}), as its shown on Fig.\ref{Fig1}  , but into \textit{hadrons}. These are particle made of quarks, all the decay channels of the tau containing hadrons in the final state are called \textit{hadronic tau decays}. An example of this decay mode is shown in Fig.\ref{Fig2}

\begin{figure}[h]
	\centering
	\includegraphics[width=0.7\textwidth]{figures/Fig1}
	\caption{Tau leptonic decay mode. Tau lepton is kinematically allowed to decay into muons or electrons, not that in this decay mode two neutrinos of different flavour are produced.}
	\label{Fig1}
\end{figure}

\begin{figure}[h]
	\centering
	\includegraphics[width=0.7\textwidth]{figures/Fig2}
	\caption{Tau hadronic decay mode. Tau lepton is kinematically allowed only to decay into hadrons containing up, down and strange quarks. This results on final states containing multiple pions or kaons \cite{Davier_2006}.}
	\label{Fig2}
\end{figure}
Naively, if we were to estimate the branching fraction for hadronic and leptonic tau decay modes, defined as:
\begin{equation}
	\beta(\tau\to X\nu_\tau)=\frac{\Gamma(\tau\to X\nu_\tau)}{\Gamma_{\text{tot}}},
\end{equation}
where X could be any group of leptons or hadrons and $\Gamma_{\text{tot}}$ is the total decay width for the tau, we could argue that the contribution from the hadronic decays triples the one for the leptonic channels. This is because in any hadronic decay, we would have to count 3 different diagrams, like the one in Fig.\ref{Fig2} because of the 3 colour possibility for the quarks.

Thus, 
\begin{align}
\beta(\tau\to l\nu_l\nu_\tau)\approx 20\%& \hspace{1cm}l=e,\mu;
\\
\beta(\tau\to X\nu_\tau)\approx 60\%& \hspace{1cm} X=\text{hadrons+neutrinos}
\end{align}
in fact, this naive estimation is not so bad. Actual values for the leptonic branching ratios are \cite{PhysRevD.98.030001}:
\begin{align}
\beta(\tau\to e\nu_e\nu_\tau)=17.82\pm 0.04\%
\\
\beta(\tau\to \mu\nu_\mu\nu_\tau)=17.39\pm 0.04\%,
\end{align}
and the small difference is due to the mass variation between the muon and the electron.

On the other hand, the hadronic decays of the tau are more varied and can contain much more particles in the final states. The vast majority of hadronic tau decays have charged or neutral pions in the final states, but more exotic decays including kaons also happen. Branching ratios for the most important tau hadronic decays are showed on Table \ref{Table1}.
\begin{table}[]
	\centering
\begin{tabular}{|c|c|}
	\hline
	Decay mode                     & Branching fraction \\ \hline
	$\pi^\pm \nu_\tau$             & 11.1 \%            \\ \hline
	$\pi^\pm \pi^0 \nu_\tau$       & 25.4\%             \\ \hline
	$\pi^\pm \geq 2\pi^0 \nu_\tau$ & 9.1\%               \\ \hline
	$3\pi^\pm \nu_\tau$            & 9.1\%               \\ \hline
	$3\pi^\pm \geq 1\pi^0 \nu_\tau$& 4.6\%               \\ \hline
	others						   & 5.5\%               \\ \hline
\end{tabular}
	\caption{Tau hadronic decay modes branching fractions.}
	\label{Table1}
\end{table}
\section{Lepton Universality}



