\chapter{Tau physics overview}\label{chap:relatedwork}
This chapter is a review of the Tau lepton properties. They include the nature of this particle, its interactions with other Standard Model (SM) particles, its main decay modes and the physics implications of the so called Lepton Universality (LU), one of the SM predictions.  

\section{The Tau Lepton}
The Tau is a spin-$\frac{1}{2}$, electrically charged particle that belongs to the same family of particles as the electron, the muon and the neutrinos, they are all called \textit{]leptons}. Leptons are elementary particles that interact only via the weak and electromagnetic interactions, for the latter case only if they have electric charge.  

The first hints for the tau existence came from experiments conducted at the Stanford Linear Accelerator Center and Lawrence Berkeley National Laboratory \cite{PhysRevLett.35.1489}. They discovered 64 events of the form:
\begin{equation}
	e^+ + e^- \to e^\pm + \mu^\mp + \geq \text{2 undetected particles},
\end{equation}
for which there was no conventional explanation at that time. Later on, it was discovered that these events came from the production of a pair of tau particles and then a its subsequent decay on one electron, a muon and four neutrinos. Events like,
\begin{equation}
e^+ + e^- \to \tau^+ \tau^- \to e^\pm + \mu^\mp + 4\nu,
\end{equation}	
were later explored to derive tau mass and spin, confirming the existence of a third generation of leptons. 

The tau mass being $1776.86 \pm 0.12$ MeV allows this lepton not only to decay into the other lighter lepton generations (\textit{leptonic tau decays}), as its shown on Fig. , but into \textit{hadrons}. These are particle made of quarks, all the decay channels of the tau containing hadrons in the final state are called \textit{hadronic tau decays}. An example of this decay mode is shown in Fig. 

\section{Lepton Universality}



